\chapter{Conclusiones}

La creación de herramientas capaces de analizar y presentar de manera gráfica grandes cantidades de datos es una de las necesidades que demanda actualmente el mercado de la tecnología. Teniendo esto en mente, el objetivo principal del proyecto se han cumplido.

La creación de una biblioteca capaz de generar representaciones gráficas de un conjunto de datos de tipo 'Big Data', se ha cumplido con éxito. Obtener los valores necesarios para generar un gráfico a partir de este tipo de datos, apoyándose en herramientas específicas para Big Data, ha sido la clave para que este proyecto funcione. La parte difícil ha sido comunicar entre si estas herramientas para funcionar como una sola, pero todos los problemas se han superado. Tener la posibilidad de utilizar más fuentes de datos, no solo de tipo CSV, sino también JSON o incluso bases de datos SQL, no formaban parte de los objetivos, pero sí se planteó como una futura mejora.

También se ha logrado el objetivo de implementar una API RESTful. Esta permite acceder a cada una de los métodos diseñados en el sistema de manera individual para obtener información sobre ello y probar su funcionamiento.

Por último, el diseño de una interfaz web para utilizar todas las funciones implementadas en el sistema, también se ha cumplido como objetivo. Se trata de una interfaz sencilla, con la accesibilidad que proporciona una web y con la eficacia de poder gestionar las funciones de manera asíncrona. Además está diseñada bajo una estructura bien jerarquizada, de manera que se pueda modificar cualquier elemento o añadir nuevos fácilmente.

Uno de los aspectos que han faltado por implementar, también por falta de tiempo, ha sido tener una gestión de usuarios de acceso al sistema e implementar un apartado de configuración de las variables del sistema, de manera que el usuario no tenga necesidad de introducirse en el código de la aplicación. 

Poder desarrollar esta aplicación supone un avance en el campo del Big Data en cuanto a la posibilidad de entender, de una manera gráfica, qué ocurre con los datos en tiempo real. Esto significa que para medianas y grandes organizaciones que generan cada día una multitud de datos, tan variados y de distintas procedencias, este tipo de aplicaciones se vuelven imprescindibles para poder tomar las las mejores decisiones. 
