\chapter*{}
\thispagestyle{empty}

\noindent\rule[-1ex]{\textwidth}{2pt}\\[4.5ex]

Yo, \textbf{Adrián Medina González}, alumno de la titulación de la \textbf{Escuela Técnica Superior
de Ingenierías Informática y de Telecomunicación de la Universidad de Granada}, con DNI 76627532L, autorizo la
ubicación de la siguiente copia de mi Trabajo Fin de Grado en la biblioteca del centro para que pueda ser
consultada por las personas que lo deseen.

\vspace{6cm}

\noindent Fdo: Adrián Medina González

\vspace{2cm}

\begin{flushright}
Granada, Septiembre de 2017 .
\end{flushright}


\chapter*{}
%\thispagestyle{empty}

\noindent\rule[-1ex]{\textwidth}{2pt}\\[4.5ex]

D. \textbf{José Manuel Benítez Sánchez}, Profesor del Departamento Ciencias de la Computación e Inteligencia Artificial de la Universidad de Granada.
%\vspace{0.5cm}

D. \textbf{Manuel Jesús Parra Royón}, Profesor del Departamento Ciencias de la Computación e Inteligencia Artificial de la Universidad de Granada.
%\vspace{0.5cm}

\textbf{Informan:}
%\vspace{0.5cm}

Que el presente trabajo, titulado \textit{\textbf{Visualización de datos en procesamiento masivo de Big Data}},
ha sido realizado bajo su supervisión por \textbf{Adrián Medina González}, y autorizamos la defensa de dicho trabajo ante el tribunal
que corresponda.

%\vspace{0.5cm}

Y para que conste, expiden y firman el presente informe en Granada, Septiembre de 2017 .

%\vspace{1cm}

\textbf{Los directores:}

\vspace{5cm}

\noindent \textbf{José Manuel Benítez Sánchez \ \ \ \ \ Manuel Jesús Parra Royón}

\chapter*{Agradecimientos}
\thispagestyle{empty}

\vspace{1cm}


Quiero agradecer a los tutores José Manuel Benítez Sánchez y Manuel Jesús Parra Royón por el tiempo y el esfuerzo invertidos en la realización de este proyecto. 

También agradezco a la ETSIIT por permitirme trabajar con parte de los servidores para poder ejecutar la aplicación. 

Por último, agradezco toda la paciencia y el apoyo mostrados durante este tiempo a mi familia, especialmente a Verónica que ha convivido diariamente con los buenos y malos momentos durante la realización del proyecto. 

